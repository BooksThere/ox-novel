\documentclass[a5j,10pt,uplatex,openright,dvipdfmx]{utbook}
\usepackage[uplatex,deluxe]{otf}
\usepackage{pxrubrica}
\usepackage{setspace}
\usepackage{lscape}
\usepackage[dvipdfmx]{graphicx}
\usepackage{plext}

\newcommand{\rotatepbox}[1]{\rotatebox{-90}{\pbox<y>{#1}}}
\newcommand{\raiserotatepbox}[1]{\raise1ex\hbox{\rotatepbox{#1}}}

\begin{document}

%%%% titlepage
\setlength{\oddsidemargin}{-0.9cm}
\setlength{\evensidemargin}{-0.9cm}
\setlength{\topmargin}{1.8in}
\setlength{\textwidth}{50zw}

\thispagestyle{empty}
\begin{landscape}
\begin{center}
\rotatepbox{\huge Sample:サンプル小説}
\end{center}
\begin{center}
\rotatepbox{\large Org-mode で記述された小説が如何様な PDF になるのか。}
\end{center}
\hspace{1ex}\begin{center}
\rotatepbox{\large 本田そこ}
\end{center}

\end{landscape}
\newpage

%% toc
\setcounter{tocdepth}{0}
\tableofcontents
\thispagestyle{empty}
%% main
\setlength{\oddsidemargin}{-1.2cm}
\setlength{\evensidemargin}{-0.6cm}
\setlength{\topmargin}{-0.5in}
\setlength{\textwidth}{50zw}
\setstretch{1.4}
\small
\twocolumn


\chapter*{第一章}
\addcontentsline{toc}{chapter}{第一章}
これより \textbf{\textgt{第一章}} を開始。

~


\hspace{1.0em}\textgt{*第一節}
\addcontentsline{toc}{section}{第一節}

~

サンプルテキスト。

~


\hspace{1.0em}\textgt{*第二節}
\addcontentsline{toc}{section}{第二節}

~

サンプルテキストその 2。

~


\chapter*{第二章}
\addcontentsline{toc}{chapter}{第二章}
これからは第二章。

~


\hspace{1.0em}\textgt{*第一節}
\addcontentsline{toc}{section}{第一節}

~

このようにして\ruby[g]{漢字}{かんじ}にルビを振ることも可能。

~

\newpage
\onecolumn
%% colophon
\setlength{\oddsidemargin}{-0.9cm}
\setlength{\evensidemargin}{-0.9cm}
\setlength{\topmargin}{1.5in}
\setlength{\textwidth}{50zw}

\begin{landscape}
\chapter*{}
\thispagestyle{empty}
\rotatepbox{{\Large Sample:サンプル小説}}

\vspace{1ex}
\rotatepbox{{\normalsize Org-mode で記述された小説が如何様な PDF になるのか。}}

\vspace{1zw}
\rotatepbox{平成 28 年 4 月 12 日 初版}

\begin{table}[htb]
\begin{tabular}{rl}
\raiserotatepbox{著者} & \raiserotatepbox{本田そこ}\\
\raiserotatepbox{発行} & \raiserotatepbox{そこそこ}
\end{tabular}
\end{table}
\end{landscape}

\end{document}
